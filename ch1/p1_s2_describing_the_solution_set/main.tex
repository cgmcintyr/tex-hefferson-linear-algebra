% Document for my solutions to Hefferson's Linear Algebra Chapter 1.I.2
\documentclass[12pt]{article}

% --------------------------------------------------------------------------- %
% Packages                                                                    %
% --------------------------------------------------------------------------- %

% Customize page layout.
\usepackage[hmargin=0.75in, vmargin=1in]{geometry}

% Use T1 font encoding (see SE answer https://tex.stackexchange.com/a/677).
\usepackage[T1]{fontenc}
\usepackage[utf8]{inputenc}

% Adds optional arg to enumerate to determine how counter is printed.
\usepackage{enumitem}

% Enable creating framed/bordered environments.
\usepackage{framed}

% Cutomize page headers
\usepackage{fancyhdr}

% Adds \pageref{LastPage} to reference last page of document.
\usepackage{lastpage}

% Loads amsmath and some very useful complements (pmatrix*)
\usepackage{mathtools}
\usepackage{amssymb, amsthm}

% Systems of linear equations
\usepackage{systeme}

% Provides conditionals
\usepackage{ifthen}

% --------------------------------------------------------------------------- %
% Configuration                                                               %
% --------------------------------------------------------------------------- %

% Reduce spacings/indents
\setlength{\topskip}{0mm}
\setlength{\parskip}{1ex}
\setlength{\parindent}{0mm}

% enumitem - disable vertical seperators in lists
%\setlist{nosep}

% fancyhdr - use fancy headers to add book's subsection and section title.
\pagestyle{fancy}
\lhead{{\bf I.2. Describing The Solution Set}}
\rhead{Chapter One: Solving Linear Systems}
\setlength{\headheight}{16pt}

% fancyhdr, lastpage - set page footer to include numbering out of last page.
\cfoot{Page \thepage\ of \pageref{LastPage}}

% --------------------------------------------------------------------------- %
% Macros/Environments                                                         %
% --------------------------------------------------------------------------- %

\newenvironment{problem}[1][default]{
  \begin{framed}\begin{minipage}{0.97\textwidth}
  \setlength{\parskip}{4mm}
  {\bf Problem #1}
}{\end{minipage}\end{framed}}

\newenvironment{abc}{\begin{enumerate}[label={\bf(\alph*)}]}{\end{enumerate}}

% Create matrix
% Usage: \m{a&b&c\\x&y&z} or \m[rr:r]{a&b&c\\x&y&z}
\newcommand\m[2][]{
	\ifthenelse{\equal {#1} {}}
		% if
		{\begin{pmatrix*}[r]#2\end{pmatrix*}}
		% else
		{\left(\begin{array}{#1}#2\end{array}\right)}
}

\newcommand\resetequation[1][1]{\setcounter{equation}{#1 - 1}}


% --------------------------------------------------------------------------- %
% Content                                                                     %
% --------------------------------------------------------------------------- %

\begin{document}

\begin{problem}[2.15]
Find the indicated entry of the matrix, if it is defined.
\[
A = \m{1 &  3 & 1 \\ 2 & -1 & 4}
\]
\end{problem}

\begin{enumerate}[label={\bf(\alph*)}]
	\item $ a_{2,1} = 2 $
	\item $ a_{1,2} = 3 $
	\item $ a_{2,2} = -1 $
	\item $ a_{3,1} = undefined $
\end{enumerate}


\begin{problem}[2.16]
Give the size of each matrix.
\[
A = \m{ 1 &  3 & 1 \\ 2 & -1 & 4 }
\]
\end{problem}

\begin{enumerate}[label=\bf(\alph*)]
	\item $ \m{1 & 0 & 4 \\ 2 & 1 & 5} $ is a $ 2 \times 3 $ matrix.
	\item $ \m{1 & 1 \\ -1 & 1 \\ 3 & -1} $ is a $ 3 \times 2 $ matrix.
	\item $ \m{5 & 10 \\ 10 & 5} $ is a $ 2 \times 2 $ matrix.
\end{enumerate}


\begin{problem}[2.17]
Perform the indicated vector operation, if it is defined.
\end{problem}

\begin{enumerate}[label=\bf(\alph*)]
	\item $ \m{2 \\ 1 \\ 1} + \m{3 \\ 0 \\ 4} = \m{5 \\ 1 \\ 5} $
	\item $ 5 \m{4 \\ -1} = \m{20 \\ -1} $
	\item $ \m{1 \\ 5 \\ 1} - \m{3 \\ 1 \\ 1} = \m{-2 \\ 4 \\ 0} $
	\item $ 7 \m{2 \\ 1} + 9 \m{3 \\ 5} = \m{14 \\ 7} + \m{21 \\ 45} = \m{35 \\ 52} $
	\item $ \m{1 \\ 2} + \m{1 \\ 2 \\ 3} = undefined $
	\item $ 6 \m{3 \\ 1 \\ 1} - 4 \m{2 \\ 0 \\ 3} + 2 \m{1 \\ 1 \\ 5} = \m{18 \\ 6 \\ 6} - \m{8 \\ 0 \\ 12} + \m{2 \\ 2 \\ 10} = \m{10 \\ 6 \\ -6} + \m{2 \\ 2 \\ 10} = \m{12 \\ 8 \\ 4} $
\end{enumerate}


\begin{problem}[2.18]
Solve each system using matrix notation. Express the solution using vectors.
\end{problem}

\begin{abc}
	% a
	\begin{item}
	\resetequation
	\begin{equation}
		\sysdelim..\systeme{3x + 6y = 18, x + 2y = 6}
	\end{equation}

	As a matrix:
	\begin{equation}
		\m[rr|r]{3 & 6 & 18 \\ 1 & 2 & 6}
	\end{equation}

	Which can be reduced as follows:
	\begin{equation}
		\m[rr|r]{3 & 6 & 18 \\ 1 & 2 & 6}
		\xrightarrow{r_1 - 3r_2}
		\m[rr|r]{0 & 0 & 0 \\ 1 & 2 & 6}
		\xrightarrow{r_1 \leftrightarrow r_2}
		\m[rr|r]{1 & 2 & 6 \\ 0 & 0 & 0}
	\end{equation}

	The solution set for this set of linear equations is:

	\begin{equation}
		\{ (x, y) \mid x + 2y = 6 \}
	\end{equation}

	As we only have one equation, the first variable $x$ is leading, and the
	second variable $y$ is free. Therefore we can define $x$ in terms of $y$
	$ x = 6 - 2y $, giving us the solutions set:

	\begin{equation}
		\{ ((6) - (2)y, y) \mid y \in \mathbb{R} \}
	\end{equation}

	Which can be rewritten using vectors by grouping together coefficients:

	\begin{equation}
		\{\m{-2 \\ 1}y + \m{6 \\ 0} \mid y \in \mathbb{R} \}
	\end{equation}

	An example solution from this set, when $y = 2$:

	\begin{equation}
		\vec{s} = \m{-2 \\ 1}2 + \m{6 \\ 0} = \m{-4 \\ 2} + \m{6 \\ 0} = \m{2 \\ 2}
	\end{equation}

	\end{item}

	% b
	\begin{item}
	\begin{equation}
		\sysdelim..\systeme{x + y = 1, x - y = -1}
	\end{equation}

	As a matrix:
	\begin{equation}
		\m[rr|r]{1 & 1 & 1 \\ 1 & -1 & -1}
	\end{equation}

	Which can be reduced as follows:
	\begin{equation}
		\m[rr|r]{1 & 1 & 1 \\ 1 & -1 & -1}
		\xrightarrow{r_1 - r_2}
		\m[rr|r]{0 & 2 & 2 \\ 1 & -1 & -1}
		\xrightarrow{r_1 \leftrightarrow \frac{1}{2}r_2}
		\m[rr|r]{1 & -1 & -1 \\ 0 & 1 & 1}
	\end{equation}

	Which gives us $y = 1$ and $x = 0$, so the solution set is:
	\begin{equation}
		\{ \m{0 \\ 1} \}
	\end{equation}
	\end{item}

	% c
	\begin{item}
	\begin{equation}
		\sysdelim..\systeme{x_1 + x_3 = 4, x_1 - x_2 + 2x_3 = 5, 4x_1 - x_2 + 5x_3 = 17}
	\end{equation}

	As a matrix:
	\begin{equation}
		\m[rrr|r]{1 & 0 & 1 & 4 \\ 1 & -1 & 2 & 5 \\ 4 & -1 & 5 & 17}
	\end{equation}

	Which can be reduced as follows:
	\begin{equation}
	\begin{aligned}
		\m[rrr|r]{1 & 0 & 1 & 4 \\ 1 & -1 & 2 & 5 \\ 4 & -1 & 5 & 17}
		\xrightarrow{r_2 - r_1}
		\m[rrr|r]{1 & 0 & 1 & 4 \\ 0 & -1 & 1 & 1 \\ 4 & -1 & 5 & 17}
		\\
		\xrightarrow{r_3 - 4r_1}
		\m[rrr|r]{1 & 0 & 1 & 4 \\ 0 & -1 & 1 & 1 \\ 0 & -1 & 1 & 1}
		\\
		\xrightarrow{r_3 - r_2}
		\m[rrr|r]{1 & 0 & 1 & 4 \\ 0 & -1 & 1 & 1 \\ 0 & 0 & 0 & 0}
	\end{aligned}
	\end{equation}

	This leaves us with $x_1$ and $x_2$ as leading variables, and $x_3$ as
	a free variable. We can rewrite equation $r_1$ as $x_1 = 4 - x_3$ and
	$r_2$ as $x_2 = x_3 - 1$, giving the solution set:

	\begin{equation}
		\{ (4 - x_3, x_3 - 1, x_3) \mid x_3 \in \mathbb{R} \}
	\end{equation}

	Written using vectors:
	\begin{equation}
		\{ \m{-1 \\ 1 \\ 1} x_3 + \m{4 \\ -1 \\ 0} \mid x_3 \in \mathbb{R} \}
	\end{equation}
	\end{item}

	% d
	\begin{item}
	\begin{equation}
		\sysdelim..\systeme{2a + b - c = 2, 2a + c = 3, a - b = 0}
	\end{equation}

	As a matrix:
	\begin{equation}
		\m[rrr|r]{2 & 1 & -1 & 2 \\ 2 & 0 & 1 & 3 \\ 1 & - 1 & 0 & 0}
	\end{equation}

	Which can be reduced as follows:
	\begin{equation}
	\begin{aligned}
		\m[rrr|r]{2 & 1 & -1 & 2 \\ 2 & 0 & 1 & 3 \\ 1 & - 1 & 0 & 0}
		\xrightarrow{r_2 - 2r_3}
		\m[rrr|r]{2 & 1 & -1 & 2 \\ 0 & 2 & 1 & 3 \\ 1 & - 1 & 0 & 0}
		\\
		\xrightarrow{r_1 - 2r_3}
		\m[rrr|r]{0 & 3 & -1 & 2 \\ 0 & 2 & 1 & 3 \\ 1 & - 1 & 0 & 0}
		\\
		\xrightarrow{r_1 \leftrightarrow r_3}
		\m[rrr|r]{1 & - 1 & 0 & 0 \\ 0 & 2 & 1 & 3 \\ 0 & 3 & -1 & 2}
		\\
		\xrightarrow{2r_3 - 3r_2}
		\m[rrr|r]{ 1 & - 1 & 0 & 0 \\ 0 & 2 & 1 & 3 \\ 0 & 0 & -5 & -5}
		\\
		\xrightarrow{-1r_3}
		\m[rrr|r]{ 1 & - 1 & 0 & 0 \\ 0 & 2 & 1 & 3 \\ 0 & 0 & 5 & 5}
	\end{aligned}
	\end{equation}

	This gives us $c = 1$, $b = 1$ and $c = 1$. The solution set for this is:
	\begin{equation}
		\{ \m{1 \\ 1 \\ 1} \}
	\end{equation}
	\end{item}
\end{abc}

\begin{problem}[2.21]
	The vector is in the set. What value of the paramaters produce that vector?
\end{problem}

\begin{abc}
	% a
	\begin{item}
	\resetequation

	\begin{equation}
	\begin{aligned}
		\m{5 \\ - 5} \in \{ \m{1 \\ -1}k \mid k \in \mathbb{R} \}
	\end{aligned}
	\end{equation}
	
	Using a system of linear equations:

	\begin{equation}
	\begin{aligned}
		\sysdelim..\systeme{k = 5, -k = -5}
	\end{aligned}
	\end{equation}

	We can see $k = 5$ results in $\m{5 \\ - 5}$.
	\end{item}

	\begin{item}

	\begin{equation}
	\begin{aligned}
		\m{-1 \\ 2 \\ 1} \in \{ \m{-2 \\ 1 \\ 0} i + \m{3 \\ 0 \\ 1}j \mid i, j \in \mathbb{R} \}
	\end{aligned}
	\end{equation}
	
	Using a system of linear equations:

	\begin{equation}
	\begin{aligned}
		\sysdelim..\systeme{-2i + 3j = -1, i = 2, j = 1}
	\end{aligned}
	\end{equation}
	
	We can see $i = 2$ and $j = 1$ result in $\m{-1 \\ 2 \\ 1}$.
	\end{item}

	\begin{item}

	\begin{equation}
	\begin{aligned}
		\m{0 \\ -4 \\ 2} \in \{ \m{1 \\ 1 \\ 0} m + \m{2 \\ 0 \\ 1}n \mid m, n \in \mathbb{R} \}
	\end{aligned}
	\end{equation}
	
	Using a system of linear equations:

	\begin{equation}
	\begin{aligned}
		\sysdelim..\systeme{m + 2n = 0, m = -4, n = 2}
	\end{aligned}
	\end{equation}
	
	We can see $m = -4$ and $n = 2$ result in $\m{0 \\ -4 \\ 2}$.
	\end{item}
\end{abc}


\begin{problem}[2.22]
	Decide if the vector is in the set.
\end{problem}

\begin{abc}
	% a
	\begin{item}
	\resetequation
	\begin{equation}
	\begin{aligned}
		\m{3 \\ -1} \in \{ \m{-6 \\ 2}k \mid k \in \mathbb{R} \}
	\end{aligned}
	\end{equation}
	
	Using a system of linear equations:

	\begin{equation}
	\begin{aligned}
		\sysdelim..\systeme{-6k = 3, 2k = -1}
	\end{aligned}
	\end{equation}

	Representing that system as a matrix:

	\begin{equation}
	\begin{aligned}
		\m[r|r]{-6 & 3 \\ 2 & -1}
		\xrightarrow{r_1 + 3r_2}
		\m[r|r]{0 & 0 \\ 2 & -1}
		\xrightarrow{\frac{1}{2}r_2}
		\m[r|r]{0 & 0 \\ 1 & -\frac{1}{2}}
		\xrightarrow{r_1 \xleftrightarrow r_2}
		\m[r|r]{1 & -\frac{1}{2} \\ 0 & 0}
	\end{aligned}
	\end{equation}

	So this vector is part of the set when $k = -\frac{1}{2}$.
	\end{item}

	% b
	\begin{item}
	\begin{equation}
	\begin{aligned}
		\m{5 \\ 4} \in \{ \m{5 \\ -4}j \mid j \in \mathbb{R} \}
	\end{aligned}
	\end{equation}
	
	Intuitively we can see that $\m{5 & 4}^T$ is not a member of this set,
	as there is no single real value $j$ that would invert $-4$ but not
	invert $5$.
	\end{item}

	% c
	\begin{item}
	\begin{equation}
	\begin{aligned}
		\m{2 \\ 1 \\ -1} \in \{\m{0 \\ 3 \\ -7} + \m{1 \\ -1 \\ 3}r \mid r \in \mathbb{R} \}
	\end{aligned}
	\end{equation}

	Using a system of linear equations:

	\begin{equation}
	\begin{aligned}
		\sysdelim..\systeme{r = 2, -r + 3 = 1, 3r + -7 = -1}
	\end{aligned}
	\end{equation}


	Which can be simplified to:

	\begin{equation}
	\begin{aligned}
		\sysdelim..\systeme{r = 2, r = 2, r = 2}
	\end{aligned}
	\end{equation}

	So this vector is a member the set, corresponding to $r = 2$.
	\end{item}

	% d
	\begin{item}
	\begin{equation}
	\begin{aligned}
		\m{1 \\ 0 \\ 1} \in \{\m{2 \\ 0 \\ 1}j + \m{-3 \\ -1 \\ 1}k \mid j, k \in \mathbb{R} \}
	\end{aligned}
	\end{equation}

	Using a system of linear equations:

	\begin{equation}
	\begin{aligned}
		\sysdelim..\systeme{2j - 3k = 1, - k = 0, j + k = 1}
	\end{aligned}
	\end{equation}

	Representing this system as a matrix:

	\begin{equation}
	\begin{aligned}
		\m[rr|r]{2 & -3 & 1 \\ 0 & -1 & 0 \\ 1 & 1 & 1}
		\xrightarrow{-1r_2}
		\m[rr|r]{2 & -3 & 1 \\ 0 & 1 & 0 \\ 1 & 1 & 1}
		\\
		\xrightarrow{r_1 - 2r_3}
		\m[rr|r]{0 & -5 & -1 \\ 0 & 1 & 0 \\ 1 & 1 & 1}
		\\
		\xrightarrow{-\frac{1}{5}r_1}
		\m[rr|r]{0 & 1 & \frac{1}{5} \\ 0 & 1 & 0 \\ 1 & 1 & 1}
		\\
		\xrightarrow{r_3 - r_2}
		\m[rr|r]{0 & 1 & \frac{1}{5} \\ 0 & 1 & 0 \\ 1 & 0 & 1}
		\\
		\xrightarrow{r_3 \xleftrightarrow r_1}
		\m[rr|r]{1 & 0 & 1 \\ 0 & 1 & 0 \\ 0 & 1 & \frac{1}{5}}
	\end{aligned}
	\end{equation}

	So this vector is \textbf{not} a member the set, as the system of linear equations is inconsistent for $\m{1 & 0 & 1}^T$.
	\end{item}
\end{abc}

\begin{problem}[2.23]
	A famer with a 1200 acre farm is considering planting three different crops.

	\begin{itemize}
		\item Corn costs \$20 per acre.
		\item Soybeans cost \$50 per acre.
		\item Oat cost \$12 per acre.
	\end{itemize}

	The famer has \$40,000 and intends to spend it all.
\end{problem}

\begin{abc}
	% a
	\begin{item}
	\resetequation
	We can represent this problem as a system of linear equations:

	\begin{equation}\label{eqn:223_a_1}
	\begin{aligned}
		\sysdelim..\systeme{20x_1 + 50x_2 + 12x_3 = 40000, x_1 + x_2 + x_3 = 1200}
	\end{aligned}
	\end{equation}
	\end{item}

	% b
	\begin{item}\label{subproblem:223_1_b}
	We can represent (\ref{eqn:223_a_1}) in matrix form:

	\begin{equation}
	\begin{aligned}
		\m[rrr|r]{20 & 50 & 12 & 40000 \\ 1 & 1 & 1 & 1200}
		\xrightarrow{r_1 - 20r_2}
		\m[rrr|r]{0 & 30 & -8 & 16000 \\ 1 & 1 & 1 & 1200}
		\\
		\xrightarrow{\frac{1}{30}r_1}
		\m[rrr|r]{0 & 1 & -\frac{4}{15} & \frac{1600}{3} \\ 1 & 1 & 1 & 1200}
	\end{aligned}
	\end{equation}

	Which makes $x_3$ our free variable, so we can rewrite $x_2$ as:
	\begin{equation}
	\begin{aligned}
		x_2 &= \frac{1600}{3} + \frac{4}{15}x_3 \\
	\end{aligned}
	\end{equation}

	And rewrite $x_1$ as:
	\begin{equation}
	\begin{aligned}
		x_1 &= 1200 -x_2 - x_3 \\
		    &= 1200 - \frac{1600}{3} - \frac{4}{15}x_3 - x_3 \\
		    &= \frac{2000}{3} - \frac{19}{15}x_3
	\end{aligned}
	\end{equation}

	Giving the solution set:
	\begin{equation}
	\begin{aligned}
		&\{(\frac{2000}{3} - \frac{19}{15}x_3, \frac{1600}{3} + \frac{4}{15}x_3, x_3) \mid x_3 \in \mathbb{R} \} \\
		= &\{ \m{- 1\frac{19}{15} \\ \frac{4}{15} \\ 1}x_3 + \m{\frac{2000}{3} \\ \frac{1600}{3} \\ 0} \mid x_3 \in \mathbb{R} \}
	\end{aligned}
	\end{equation}

	This gives us many possible solutions such as when $x_3 = 0$:

	\begin{equation}
	\begin{aligned}
		\vec{s}_1 = \m{0 \\ 0 \\ 0} + \m{\frac{2000}{3} \\ \frac{1600}{3} \\ 0} = \m{\frac{2000}{3} \\ \frac{1600}{3} \\ 0}
	\end{aligned}
	\end{equation}

	Or when $x_3 = 526$:
	\begin{equation}
	\begin{aligned}
		\vec{s}_2 = \m{\frac{2}{5} \\ \frac{3368}{5} \\ 526}
	\end{aligned}
	\end{equation}
	\end{item}

	% c
	\begin{item}
	Supposing the farmer can bring in:

	\begin{itemize}
		\item \$100 per acre for Corn.
		\item \$300 per acre for Soybeans.
		\item \$80 per acre for Oats.
	\end{itemize}

	We can check which of the results from \ref{subproblem:223_1_b} is most 
	profitable by plugging the solution sets into:

	\begin{equation}
	\begin{aligned}
		100x_1 + 300x_2 + 80x_3
	\end{aligned}
	\end{equation}

	For $\vec{s}_1$, the farmer's revenue is $\sim\$226666$.

	For $\vec{s}_2$, the farmer's revenue is $\$244200$.

	\end{item}
\end{abc}

\begin{problem}[2.25]
	Using Gauss's Method
\end{problem}


\begin{abc}
\resetequation
\begin{item}
	Solve the left-hand side of:
	\begin{equation}
		\sysdelim..\systeme{w + 2x - z = a, 2w + y = b, w + x + 2z = -2}
	\end{equation}

	As a matrix:
	\begin{equation}
	\begin{aligned}
		\m[rrrr|r]{
			1 & 2 & 0 & -1 & a \\
			2 & 0 & 1 & 0 & b \\
			1 & 1 & 0 & 2 & c
		}
		\xrightarrow{r_1 - r_3}
		\m[rrrr|r]{
			0 & 1 & 0 & -3 & a - c \\
			2 & 0 & 1 & 0 & b \\
			1 & 1 & 0 & 2 & c
		}
		\\
		\xrightarrow{r_2 - 2r_3}
		\m[rrrr|r]{
			0 & 1 & 0 & -3 & a - c \\
			0 & -2 & 1 & -4 & b - 2c \\
			1 & 1 & 0 & 2 & c
		}
		\\
		\xrightarrow{r_2 + 2r_1}
		\m[rrrr|r]{
			0 & 1 & 0 & -3 & a - c \\
			0 & 0 & 1 & -10 & 2a + b - 4c \\
			1 & 1 & 0 & 2 & c
		}
		\\
		\xrightarrow{r_3 - r_1}
		\m[rrrr|r]{
			0 & 1 & 0 & -3 & a - c \\
			0 & 0 & 1 & -10 & 2a + b - 4c \\
			1 & 0 & 0 & 5 & -a + 2c
		}
	\end{aligned}
	\end{equation}

	Which gives us:

	\begin{equation}
	\begin{aligned}
		w &= - a + 2c + -5z \\ % x
		x &= a - c + 3z \\     % y
		y &= 2a + b - 4c + 10z   % z
	\end{aligned}
	\end{equation}

	Therefore our solution set is:

	\begin{equation}
	\begin{aligned}
		\{ \m{-a + 2c \\ a - c \\ 2a + b - 4c \\ 0} + \m{-5 \\ 3 \\ 10 \\ 1}w \mid w \in \mathbb{R} \}
	\end{aligned}
	\end{equation}


\end{item}
\end{abc}


\end{document}
