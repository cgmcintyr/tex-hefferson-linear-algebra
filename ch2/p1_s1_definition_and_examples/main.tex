% Document for my solutions to Hefferson's Linear Algebra Chapter 1.I.2
\documentclass[12pt]{article}

% --------------------------------------------------------------------------- %
% Packages                                                                    %
% --------------------------------------------------------------------------- %

% Customize page layout.
\usepackage[hmargin=0.75in, vmargin=1in]{geometry}

% Use T1 font encoding (see SE answer https://tex.stackexchange.com/a/677).
\usepackage[T1]{fontenc}
\usepackage[utf8]{inputenc}

% Adds optional arg to enumerate to determine how counter is printed.
\usepackage{enumitem}

% Enable creating framed/bordered environments.
\usepackage{framed}

% Cutomize page headers
\usepackage{fancyhdr}

% Adds \pageref{LastPage} to reference last page of document.
\usepackage{lastpage}

% Loads amsmath and some very useful complements (pmatrix*)
\usepackage{mathtools}
\usepackage{amssymb, amsthm}

% Systems of linear equations
\usepackage{systeme}

% Provides conditionals
\usepackage{ifthen}

% --------------------------------------------------------------------------- %
% Configuration                                                               %
% --------------------------------------------------------------------------- %

% Reduce spacings/indents
\setlength{\topskip}{0mm}
\setlength{\parskip}{1ex}
\setlength{\parindent}{0mm}

% enumitem - disable vertical seperators in lists
%\setlist{nosep}

% fancyhdr - use fancy headers to add book's subsection and section title.
\pagestyle{fancy}
\lhead{{\bf II.1. Defintion And Examples}}
\rhead{Chapter Two: Vector Spaces}
\setlength{\headheight}{16pt}

% fancyhdr, lastpage - set page footer to include numbering out of last page.
\cfoot{Page \thepage\ of \pageref{LastPage}}

% --------------------------------------------------------------------------- %
% Macros/Environments                                                         %
% --------------------------------------------------------------------------- %

\newenvironment{problem}[1][default]{
  \begin{framed}\begin{minipage}{0.97\textwidth}
  \setlength{\parskip}{4mm}
  {\bf Problem #1}
}{\end{minipage}\end{framed}}

\newenvironment{abc}{\begin{enumerate}[label={\bf(\alph*)}]}{\end{enumerate}}

% Create matrix
% Usage: \m{a&b&c\\x&y&z} or \m[rr:r]{a&b&c\\x&y&z}
\newcommand\m[2][]{
	\ifthenelse{\equal {#1} {}}
		% if
		{\begin{pmatrix*}[r]#2\end{pmatrix*}}
		% else
		{\left(\begin{array}{#1}#2\end{array}\right)}
}

\newcommand\resetequation[1][1]{\setcounter{equation}{#1 - 1}}
\renewcommand{\vec}[1]{\mathbf{#1}}


% --------------------------------------------------------------------------- %
% Content                                                                     %
% --------------------------------------------------------------------------- %

\begin{document}

\begin{problem}[1.17]
Name the zero vector for each of these vector spaces.
\end{problem}

\begin{abc}
\begin{item}
	The space of three polynomials under the natural operations
	(i.e. $a_0x^0 + a_1x^1 + a_2x^2 + a_3x^3$) has zero vector
	$0x^0 + 0x^1 + 0x^2 + 0x^3$.
\end{item}

\begin{item}
	The space of $2 \times 3$ matrices $\m{ a_{1,1} & a_{1,2} & a_{1,3} \\ a_{2,1} & a_{2,2} & a_{2,3}}$
	has the zero vector $\m{0 & 0 & 0 \\ 0 & 0 & 0}$.
\end{item}

\begin{item}
	The space of $\{f: [0..1] \rightarrow \mathbb{R} \mid f \; \text{is continuous}\}$ has the constant function $f(x) = 0$ as its zero vector.
\end{item}

\begin{item}
	The space of functions $f: \mathbb{N} \rightarrow \mathbb{R}$ has the constant function $f(n) = 0$ as its zero vector.
\end{item}
\end{abc}

\begin{problem}[1.18]
	Find the additive inverse, in the vector space, of the vector.
\end{problem}

\begin{abc}
\begin{item}
	In $\mathcal{P}_3$, the vector $\vec{v} = -3 - 2x + x^2$.

	\begin{equation*}
	\begin{aligned}
		\vec{v} + \vec{v^{-1}} &= \vec{0} \\
		\m{-3 \\ -2 \\ 1} + \vec{v^{-1}} &= \m{0 \\ 0 \\0} \\
		\vec{v^{-1}} &= \m{0 \\ 0 \\0} - \m{-3 \\ -2 \\ 1} \\
		\vec{v^{-1}} &= \m{3 \\ 2 \\ -1}
	\end{aligned}
	\end{equation*}
\end{item}

\begin{item}
	In the $2 \times 2$ space, vector $\vec{v} = \m{1 & -1 \\ 0 & 3}$.

	\begin{equation*}
	\begin{aligned}
		\vec{v} + \vec{v^{-1}} &= \vec{0} \\
		\m{1 & -1 \\ 0 & 3} + \vec{v^{-1}} &= \m{0 & 0 \\ 0 & 0} \\
		\vec{v^{-1}} &= \m{0 & 0 \\ 0 & 0} - \m{1 & -1 \\ 0 & 3} \\
		\vec{v^{-1}} &= \m{-1 & 1 \\ 0 & -3}
	\end{aligned}
	\end{equation*}
\end{item}

\begin{item}
	In the $\{ae^x + be^{-x} \mid a, b \in \mathbb{R} \}$, the space of real variable $x$ under the natrual operations, the vector $\vec{v} = 3e^x - 2e^{-x}$


	Here the zero vector $\vec{0} = \m{0 \\ 0}$.

	So:

	\begin{equation*}
	\begin{aligned}
		\vec{v} + \vec{v^{-1}} &= \vec{0} \\
		\m{3 \\ -2} + \vec{v^{-1}} &= \m{0 \\ 0} \\
		\vec{v^{-1}} &= \m{0 \\ 0} - \m{3 \\ -2} \\
		\vec{v^{-1}} &= \m{-3 \\ 2}
	\end{aligned}
	\end{equation*}
\end{item}
\end{abc}

\end{document}
