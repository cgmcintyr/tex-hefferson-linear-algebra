% Document for my solutions to Hefferson's Linear Algebra Chapter 1.I.2
\documentclass[12pt]{article}

% --------------------------------------------------------------------------- %
% Packages                                                                    %
% --------------------------------------------------------------------------- %

% Customize page layout.
\usepackage[hmargin=0.75in, vmargin=1in]{geometry}

% Use T1 font encoding (see SE answer https://tex.stackexchange.com/a/677).
\usepackage[T1]{fontenc}

% Adds optional arg to enumerate to determine how counter is printed.
\usepackage{enumitem}

% Enable creating framed/bordered environments.
\usepackage{framed}

% Cutomize page headers
\usepackage{fancyhdr}

% Adds \pageref{LastPage} to reference last page of document.
\usepackage{lastpage}

% Loads amsmath and some very useful complements (pmatrix*)
\usepackage{mathtools}

% Systems of linear equations
\usepackage{systeme}

% Provides conditionals
\usepackage{ifthen}

% --------------------------------------------------------------------------- %
% Configuration                                                               %
% --------------------------------------------------------------------------- %

% Reduce spacings/indents
\setlength{\topskip}{0mm}
\setlength{\parskip}{1ex}
\setlength{\parindent}{0mm}

% enumitem - disable vertical seperators in lists
%\setlist{nosep}

% fancyhdr - use fancy headers to add book's subsection and section title.
\pagestyle{fancy}
\lhead{{\bf I.2. Describing The Solution Set}}
\rhead{Chapter One: Solving Linear Systems}

% fancyhdr, lastpage - set page footer to include numbering out of last page.
\cfoot{Page \thepage\ of \pageref{LastPage}}

% --------------------------------------------------------------------------- %
% Macros/Environments                                                         %
% --------------------------------------------------------------------------- %

\newenvironment{problem}[1][default]{
  \begin{framed}\begin{minipage}{0.97\textwidth}
  \setlength{\parskip}{4mm}
  {\bf Problem #1}
}{\end{minipage}\end{framed}}

\newenvironment{abc}{\begin{enumerate}[label={\bf(\alph*)}]}{\end{enumerate}}

% Create matrix
% Usage: \m{a&b&c\\x&y&z} or \m[rr:r]{a&b&c\\x&y&z}
\newcommand\m[2][]{
	\ifthenelse{\equal {#1} {}}
		% if
		{\begin{pmatrix*}[r]#2\end{pmatrix*}}
		% else
		{\left(\begin{array}{#1}#2\end{array}\right)}
}

\newcommand\resetequation[1][1]{\setcounter{equation}{#1 - 1}}


% --------------------------------------------------------------------------- %
% Content                                                                     %
% --------------------------------------------------------------------------- %

\begin{document}

\begin{problem}[2.15]
Find the indicated entry of the matrix, if it is defined.
\[
A = \m{1 &  3 & 1 \\ 2 & -1 & 4}
\]
\end{problem}

\begin{enumerate}[label={\bf(\alph*)}]
	\item $ a_{2,1} = 2 $
	\item $ a_{1,2} = 3 $
	\item $ a_{2,2} = -1 $
	\item $ a_{3,1} = undefined $
\end{enumerate}


\begin{problem}[2.16]
Give the size of each matrix.
\[
A = \m{ 1 &  3 & 1 \\ 2 & -1 & 4 }
\]
\end{problem}

\begin{enumerate}[label=\bf(\alph*)]
	\item $ \m{1 & 0 & 4 \\ 2 & 1 & 5} $ is a $ 2 \times 3 $ matrix.
	\item $ \m{1 & 1 \\ -1 & 1 \\ 3 & -1} $ is a $ 3 \times 2 $ matrix.
	\item $ \m{5 & 10 \\ 10 & 5} $ is a $ 2 \times 2 $ matrix.
\end{enumerate}


\begin{problem}[2.17]
Perform the indicated vector operation, if it is defined.
\end{problem}

\begin{enumerate}[label=\bf(\alph*)]
	\item $ \m{2 \\ 1 \\ 1} + \m{3 \\ 0 \\ 4} = \m{5 \\ 1 \\ 5} $
	\item $ 5 \m{4 \\ -1} = \m{20 \\ -1} $
	\item $ \m{1 \\ 5 \\ 1} - \m{3 \\ 1 \\ 1} = \m{-2 \\ 4 \\ 0} $
	\item $ 7 \m{2 \\ 1} + 9 \m{3 \\ 5} = \m{14 \\ 7} + \m{21 \\ 45} = \m{35 \\ 52} $
	\item $ \m{1 \\ 2} + \m{1 \\ 2 \\ 3} = undefined $
	\item $ 6 \m{3 \\ 1 \\ 1} - 4 \m{2 \\ 0 \\ 3} + 2 \m{1 \\ 1 \\ 5} = \m{18 \\ 6 \\ 6} - \m{8 \\ 0 \\ 12} + \m{2 \\ 2 \\ 10} = \m{10 \\ 6 \\ -6} + \m{2 \\ 2 \\ 10} = \m{12 \\ 8 \\ 4} $
\end{enumerate}


\begin{problem}[2.18]
Solve each system using matrix notation. Express the solution using vectors.
\end{problem}

\begin{abc}
	\begin{item}
	\resetequation
	\begin{equation}
		\sysdelim..\systeme{3x + 6y = 18, x + 2y = 6}
	\end{equation}
	\end{item}

	As a matrix:
	\begin{equation}
		\m[rr|r]{3 & 6 & 18 \\ 1 & 2 & 6}
	\end{equation}

	Which can be reduced:
	\begin{equation}
		\m[rr|r]{3 & 6 & 18 \\ 1 & 2 & 6}
		\xrightarrow{r_1 - 3r_2}
		\m[rr|r]{0 & 0 & 0 \\ 1 & 2 & 6}
		\xrightarrow{r_1 \leftrightarrow r_2}
		\m[rr|r]{1 & 2 & 6 \\ 0 & 0 & 0}
	\end{equation}
\end{abc}

\end{document}
