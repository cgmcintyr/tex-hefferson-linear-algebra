% Document for my solutions to Hefferson's Linear Algebra Chapter 1.I.2
\documentclass[12pt]{article}

% --------------------------------------------------------------------------- %
% Packages                                                                    %
% --------------------------------------------------------------------------- %

% Customize page layout.
\usepackage[hmargin=0.75in, vmargin=1in]{geometry}

% Use T1 font encoding (see SE answer https://tex.stackexchange.com/a/677).
\usepackage[T1]{fontenc}
\usepackage[utf8]{inputenc}

% Adds optional arg to enumerate to determine how counter is printed.
\usepackage{enumitem}

% Enable creating framed/bordered environments.
\usepackage{framed}

% Cutomize page headers
\usepackage{fancyhdr}

% Adds \pageref{LastPage} to reference last page of document.
\usepackage{lastpage}

% Loads amsmath and some very useful complements (pmatrix*)
\usepackage{mathtools}
\usepackage{amssymb, amsthm}

% Systems of linear equations
\usepackage{systeme}

% Provides conditionals
\usepackage{ifthen}

% Degrees
\usepackage{textcomp}
\usepackage{gensymb}


% --------------------------------------------------------------------------- %
% Configuration                                                               %
% --------------------------------------------------------------------------- %

% Reduce spacings/indents
\setlength{\topskip}{0mm}
\setlength{\parskip}{1ex}
\setlength{\parindent}{0mm}

% enumitem - disable vertical seperators in lists
%\setlist{nosep}

% fancyhdr - use fancy headers to add book's subsection and section title.
\pagestyle{fancy}
\lhead{{\bf II.2. Length and Angle Measures}}
\rhead{Chapter One: Linear Systems}
\setlength{\headheight}{16pt}

% fancyhdr, lastpage - set page footer to include numbering out of last page.
\cfoot{Page \thepage\ of \pageref{LastPage}}

% --------------------------------------------------------------------------- %
% Macros/Environments                                                         %
% --------------------------------------------------------------------------- %

\newenvironment{problem}[1][default]{
  \begin{framed}\begin{minipage}{0.97\textwidth}
  \setlength{\parskip}{4mm}
  {\bf Problem #1}
}{\end{minipage}\end{framed}}

\newenvironment{abc}{\begin{enumerate}[label={\bf(\alph*)}]}{\end{enumerate}}

% Create matrix
% Usage: \m{a&b&c\\x&y&z} or \m[rr:r]{a&b&c\\x&y&z}
\newcommand\m[2][]{
	\ifthenelse{\equal {#1} {}}
		% if
		{\begin{pmatrix*}[r]#2\end{pmatrix*}}
		% else
		{\left(\begin{array}{#1}#2\end{array}\right)}
}

\newcommand\magnitude[1]{
	\lvert{#1}\rvert
}

% Restyle vectors to be bold
\renewcommand{\vec}[1]{\mathbf{#1}}

\newcommand\resetequation[1][1]{\setcounter{equation}{#1 - 1}}


% --------------------------------------------------------------------------- %
% Content                                                                     %
% --------------------------------------------------------------------------- %

\begin{document}


\begin{problem}[2.11]
	Find the length of each vector
\end{problem}

\begin{abc}
\item{
	\begin{equation*}
	\begin{aligned}
		\vec{a} &= \m{3 \\ 1} \\
		\magnitude{\vec{a}} &= \sqrt{\magnitude{\vec{a}}^2} \\
		                    &= \sqrt{\vec{a} \cdot \vec{a}} \\
		                    &= \sqrt{3^2 + 1 ^2} \\
		                    &= \sqrt{10}
	\end{aligned}
	\end{equation*}
}

\item{
	\begin{equation*}
	\begin{aligned}
		\vec{b} &= \m{-1 \\ 2} \\
		\magnitude{\vec{b}} &= \sqrt{-1^2 + 2^2} \\
		                    &= \sqrt{5}
	\end{aligned}
	\end{equation*}
}

\item{
	\begin{equation*}
	\begin{aligned}
		\vec{c} &= \m{4 \\ 1 \\ 1} \\
		\magnitude{\vec{c}} &= \sqrt{4^2 + 1^2 + 1^2} \\
		                    &= \sqrt{18}
	\end{aligned}
	\end{equation*}
}

\item{
	\begin{equation*}
	\begin{aligned}
		\vec{d} &= \m{0 \\ 0 \\ 0} \\
		\magnitude{\vec{c}} &= 0
	\end{aligned}
	\end{equation*}
}
\end{abc}

\begin{problem}[2.12]
	Find the angle between the following pairs of vectors.
\end{problem}

\begin{abc}
\item{
	\begin{equation*}
	\begin{aligned}
		\vec{a} &= \m{1 \\ 2} \quad \vec{b} = \m{1 \\ 4} \\
		\theta_{a,b} &= arccos \left( \frac{\vec{a}\cdot\vec{b}}{\magnitude{\vec{a}} \magnitude{\vec{b}}} \right) \\
		\vec{a} \cdot \vec{b} &= 1 \cdot 1 + 2 \cdot 4 = 9 \\
		\magnitude{\vec{a}} &= \sqrt{\vec{a} \cdot \vec{a}} = \sqrt{5} \\
		\magnitude{\vec{b}} &= \sqrt{\vec{b} \cdot \vec{b}} = \sqrt{17} \\
		\theta_{a,b} &= arccos \left( \frac{9}{\sqrt{5} \cdot \sqrt{17}} \right) \\
			     &\approx 0.22 \: \text{radians}
	\end{aligned}
	\end{equation*}
}

\item{
	\begin{equation*}
	\begin{aligned}
		\vec{a} &= \m{1 \\ 2 \\ 0} \quad \vec{b} = \m{0 \\ 4 \\ 1} \\
		\theta_{a,b} &= arccos \left( \frac{\vec{a}\cdot\vec{b}}{\magnitude{\vec{a}} \magnitude{\vec{b}}} \right) \\
		\vec{a} \cdot \vec{b} &= 1 \cdot 0 + 2 \cdot 4 + 0 \cdot 1 = 8 \\
		\magnitude{\vec{a}} &= \sqrt{\vec{a} \cdot \vec{a}} = \sqrt{5} \\
		\magnitude{\vec{b}} &= \sqrt{\vec{b} \cdot \vec{b}} = \sqrt{17} \\
		\theta_{a,b} &= arccos \left( \frac{8}{\sqrt{5} \cdot \sqrt{17}} \right) \\
			     &\approx 0.52 \: \text{radians}
	\end{aligned}
	\end{equation*}
}

\item{
	The angle between vectors of different sizes is not defined.

	One would have to map the components of the lower dimension vector to the dimensions of the higher dimension vector, zeroing out the others.
}
\end{abc}

\begin{problem}[2.13]
	A ship moves 1.2 miles north, 6.1 miles 38 degrees east of south, 4.0 
	miles at 89 degrees east of north, and 6.5 miles at 31 degrees east of north. 
	Find the distance between the starting and ending positions (ignore the earth's 
	curvature).
\end{problem}

Let positive $x$ values for vectors represent miles east, and negative $x$ values represent miles west.

Let positive $y$ values for vectors represent miles north, and negative $y$ values represent miles south.

The ship's movements can be described by four displacement vectors, $\vec{x_1}$, $\vec{x_2}$, $\vec{x_3}$, and $\vec{x_4}$.

The first movement $\vec{x_1}$ is the simplest to calculate:

\begin{equation}
	\vec{x_1} = \m{0 \\ 1.2}
\end{equation}

The second movement described by $\vec{x_2}$ is 38 degrees east of south, which is a 
-52 degree angle from the positive $x$ axis. Using trigonometry we find the $x$ 
component of $\vec{x_2}$ is $6.1 \cdot cos(-52\degree)$, and the $y$ component of 
$\vec{x_2}$ is $\cdot 6.1 \cdot sin(-52\degree)$. Therefore:

\begin{equation}
	\vec{x_2} = 6.1 \cdot \m{cos(-52\degree) \\ sin(-52\degree)} = \m{3.75553499949 \\ -4.806865597}
\end{equation}

The third movement is 89 degrees east of north, which is a 1 degree angle from 
the positive $x$ axis. Therefore:

\begin{equation}
	\vec{x_3} = 4.0 \cdot \m{cos(1\degree) \\ sin(1\degree)} = \m{3.99939078063 \\ 0.06980962574}
\end{equation}

The fourth movement is 59 degrees from the positive $x$ axis. Therefore:

\begin{equation}
	\vec{x_4} = 6.5 \cdot \m{cos(59\degree) \\ sin(59\degree)} = \m{3.34774748692 \\ 5.57158745456}
\end{equation}

The combining these movements into the total displacement vector $\vec{x}$:

\begin{equation}
	\vec{x} = \sum^{4}_{i=1} \vec{x_i} = \m{11.102673267 \\ 2.0345314833}
\end{equation}

Therefore:

\begin{equation}
	\text{distance moved} = \sqrt{\vec{x} \cdot \vec{x}} \approx 11.2875
\end{equation}

\begin{problem}[2.14]
	Find $k$ so that these two vectors are prependicular:
	\begin{equation*}
		\m{k \\ 1} \qquad \m{4 \\ 3}
	\end{equation*}
\end{problem}

Using the cosine rule we know that for two vectors $\vec{a}, \vec{b}$:

\begin{equation}
	cos(\theta) = \frac{\vec{a} \cdot \vec{b}}{\magnitude{\vec{a}} \magnitude{\vec{b}}}
\end{equation}

Solving for k when $\theta = 90$:

\begin{equation}
\begin{aligned}
	cos(90) &= \frac{\vec{a} \cdot \vec{b}}{\magnitude{\vec{a}} \magnitude{\vec{b}}} \\
	0 &= \frac{\vec{a} \cdot \vec{b}}{\magnitude{\vec{a}} \magnitude{\vec{b}}} \\
	0 &= \vec{a} \cdot \vec{b} \\
	0 &= 4 \dot k + 1 \cdot 3 \\
	0 &= 4k + 3 \\
	k &= -\frac{3}{4}
\end{aligned}
\end{equation}

\begin

\end{document}
