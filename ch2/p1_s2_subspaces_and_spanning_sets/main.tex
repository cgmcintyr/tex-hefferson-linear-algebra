% Document for my solutions to Hefferson's Linear Algebra Chapter 1.I.2
\documentclass[12pt]{article}

% --------------------------------------------------------------------------- %
% Packages                                                                    %
% --------------------------------------------------------------------------- %

% Customize page layout.
\usepackage[hmargin=0.75in, vmargin=1in]{geometry}

% Use T1 font encoding (see SE answer https://tex.stackexchange.com/a/677).
\usepackage[T1]{fontenc}
\usepackage[utf8]{inputenc}

% Adds optional arg to enumerate to determine how counter is printed.
\usepackage{enumitem}

% Enable creating framed/bordered environments.
\usepackage{framed}

% Cutomize page headers
\usepackage{fancyhdr}

% Adds \pageref{LastPage} to reference last page of document.
\usepackage{lastpage}

% Loads amsmath and some very useful complements (pmatrix*)
\usepackage{mathtools}
\usepackage{amssymb, amsthm}

% Systems of linear equations
\usepackage{systeme}

% Provides conditionals
\usepackage{ifthen}

% --------------------------------------------------------------------------- %
% Configuration                                                               %
% --------------------------------------------------------------------------- %

% Reduce spacings/indents
\setlength{\topskip}{0mm}
\setlength{\parskip}{1ex}
\setlength{\parindent}{0mm}

% enumitem - disable vertical seperators in lists
%\setlist{nosep}

% fancyhdr - use fancy headers to add book's subsection and section title.
\pagestyle{fancy}
\lhead{{\bf II.2. Subspaces and Spanning Sets}}
\rhead{Chapter Two: Vector Spaces}
\setlength{\headheight}{16pt}

% fancyhdr, lastpage - set page footer to include numbering out of last page.
\cfoot{Page \thepage\ of \pageref{LastPage}}

% --------------------------------------------------------------------------- %
% Macros/Environments                                                         %
% --------------------------------------------------------------------------- %

\newenvironment{problem}[1][default]{
  \begin{framed}\begin{minipage}{0.97\textwidth}
  \setlength{\parskip}{4mm}
  {\bf Problem #1}
}{\end{minipage}\end{framed}}

\newenvironment{abc}{\begin{enumerate}[label={\bf(\alph*)}]}{\end{enumerate}}

% Create matrix
% Usage: \m{a&b&c\\x&y&z} or \m[rr:r]{a&b&c\\x&y&z}
\newcommand\m[2][]{
	\ifthenelse{\equal {#1} {}}
		% if
		{\begin{pmatrix*}[r]#2\end{pmatrix*}}
		% else
		{\left(\begin{array}{#1}#2\end{array}\right)}
}

\newcommand\resetequation[1][1]{\setcounter{equation}{#1 - 1}}
\renewcommand{\vec}[1]{\mathbf{#1}}


% --------------------------------------------------------------------------- %
% Content                                                                     %
% --------------------------------------------------------------------------- %

\begin{document}

\begin{problem}[1.20]
	Which of these subsets of the vector space of $2 \times 2$ matrices are 
	subspaces dunder the inherited operations? For each one that is a 
	subspace, parametrize its desciprtion. For each that is not, ive a 
	condition that fails.
\end{problem}

\begin{abc}
\begin{item}
	Let $S = \{ \m{a & 0 \\ 0 & b} \mid a, b \in \mathbb{R} \}$.

	First we parametrize $S$:

	\begin{equation}
		\label{eqn:parametrized_S}
		S = \{ a \m{1 & 0 \\ 0 & 0} + b \m{0 & 0 \\ 0 & 1} \mid a, b \in \mathbb{R} \}
	\end{equation}

	Which shows $S$ can be described as a linear combination of $2 \times 2$ matrices.

	First we show $\vec{0} \in S$:

	\begin{equation}
	\begin{aligned}
		a &= 0 \\
		b &= 0 \\
		0 & \m{1 & 0 \\ 0 & 0} + 0 \m{0 & 0 \\ 0 & 1} = \m{0 & 0 \\ 0 & 0} = \vec{0}
	\end{aligned}
	\end{equation}

	Then we show $S$ is closed under scalar multiplication:

	\begin{equation}
	\begin{aligned}
		\vec{s} &= a \m{1 & 0 \\ 0 & 0} + b \m{0 & 0 \\ 0 & 1} \\
		r \vec{s} &= r \left( a \m{1 & 0 \\ 0 & 0} + b \m{0 & 0 \\ 0 & 1} \right) \\
		          &= r a \m{1 & 0 \\ 0 & 0} + r b \m{0 & 0 \\ 0 & 1} \\
		ra, rb &\in \mathbb{R} \\
		\therefore r \vec{s} &\in S
	\end{aligned}
	\end{equation}

	Finally we show $S$ is closed under addition:

	\begin{equation}
	\begin{aligned}
		\vec{s_1} &= a_1 \m{1 & 0 \\ 0 & 0} + b_1 \m{0 & 0 \\ 0 & 1} \\
		\vec{s_2} &= a_2 \m{1 & 0 \\ 0 & 0} + b_2 \m{0 & 0 \\ 0 & 1} \\
		\vec {s_1} + \vec{s_2} &= a_1 \m{1 & 0 \\ 0 & 0} + a_2 \m{1 & 0 \\ 0 & 0} + b_1 \m{0 & 0 \\ 0 & 1} + b_2 \m{0 & 0 \\ 0 & 1} \\
		                       &= (a_1 + a_2) \m{1 & 0 \\ 0 & 0} + (b_1 + b_2) \m{0 & 0 \\ 0 & 1} \\
		a_1 + a_2, b_1 + b_2 &\in \mathbb{R} \\
		\therefore \vec{s_1} + \vec{s_2} &\in S
	\end{aligned}
	\end{equation}

	Therefore $S$ is a subspace.

	Alternativley, using (\ref{eqn:parametrized_S}) and the definition of a spanning set

	\begin{equation}
	\begin{aligned}
		span(V) = \{ c_1 \vec{v_1} + \ldots c_n \vec{v_n} \mid c_1, \ldots, c_n \in \mathbb{R} \: \text{and} \: \vec{v_1}, \ldots, \vec{v_n} \in V \}
	\end{aligned}
	\end{equation}

	We can see that $S = span\left( \left\{ \m{1 & 0 \\ 0 & 0}, \m{0 & 0 \\ 0 & 1} \right\} \right)$,
	and as all spanning sets are subspaces, this is another proof $S$ is a subspace.
\end{item}

\begin{item}
	\resetequation
	Let $S = \{ \m{a & 0 \\ 0 & b} \mid a + b = 0 \}$.

	Paramaterizing $S$:
	
	\begin{equation}
		S = \{ \m{-b & 0 \\ 0 & b} \mid b \in \mathbb{R} \} = \{ b \m{-1 & 0 \\ 0 & 1} \mid b \in \mathbb{R} \}
	\end{equation}

	Showing $S = span\left( \left\{ \m{-1 & 0 \\ 0 & 1} \right\} \right)$, and therefore $S$ is a subspace.
\end{item}

\begin{item}
	\resetequation
	Let $S = \{ \m{a & 0 \\ 0 & b} \mid a + b = 5 \}$.

	By paramaterizing $S$:

	\begin{equation}
		S = \{ \m{ -b + 5 & 0 \\ 0 & b} \mid b \in \mathbb{R} \}
	\end{equation}
	
	We can see $S$ is not a vector space as $ \vec{0} \notin S$, which
	means $S$ is not closed under scalar multiplcation.
\end{item}

\begin{item}
	\resetequation
	Let $S = \{ \m{a & c \\ 0 & b} \mid a + b = 0, c \in \mathbb{R} \}$.

	By paramaterizing $S$:

	\begin{equation}
	\begin{aligned}
		S &= \{ \m{ -b & c \\ 0 & b} \mid b, c \in \mathbb{R} \} \\
		  &= \{ b \m{ -1 & 0 \\ 0 & 1} + c \m { 0 & 1 \\ 0 & 0 } \mid b, c \in \mathbb{R} \}
	\end{aligned}
	\end{equation}
	
	Showing $S = span\left( \left\{ \m{-1 & 0 \\ 0 & 1}, \m{0 & 1 \\ 0 & 0} \right\} \right)$, and therefore $S$ is a subspace.
\end{item}
\end{abc}


\end{document}
